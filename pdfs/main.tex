\documentclass{article}
\usepackage[T1]{fontenc}
\usepackage[polish]{babel}
\usepackage[utf8]{inputenc}
\usepackage{amsfonts}
\usepackage{mathtools}

\makeatletter
\def\alignedspace@left{\null\,}
\makeatother
\newtheorem{theorem}{Theorem}
\begin{document}

\section{Sformułowanie silne}
$$
\begin{gathered}
    \frac{d}{d x}\left(-k(x) \frac{d u(x)}{d x}\right)=0 \\
    u(2)=0 \\
    \frac{d u(0)}{d x}+u(0)=20 \\
k(x)= \begin{cases}1 & \text { dla } x \in[0,1] \\
    2 & \text { dla } x \in(1,2]\end{cases}
\end{gathered}
$$

Gdzie $u$ to poszukiwana funkcja
$$
[0,2] \ni x \rightarrow u(x) \in \mathbb{R}
$$

\section{Sformułowanie wariacyjne}
$$
\begin{gathered}
-\frac{dk}{dx}\frac{du}{dx} - k\frac{d^2u}{dx^2} = 0\\
- k\frac{d^2u}{dx^2} = 0\\
-\int_0^2k(x)\frac{d^2u}{dx^2} = 0\\
-u'(x)v(x)k(x)\big\rvert^2_0 + \int_0^2u'v'dx = 0\\
u'(0)v(0) - 2u'(2)v(2) + \int_0^2k(x)u'v'dx = 0\\
u'(0)v(0) + \int_0^2k(x)u'v'dx = 0\\
v(0)(20 - u(0)) + \int_0^2k(x)u'v'dx = 0\\
 u(0)v(0) -\int_0^2k(x)u'v'dx= 20v(0)\\
B(u,v ) = u(0)v(0) - \int_0^2k(x)u'v'dx, \text{ } L(v) = 20v(0)\\
B(u, v) = L(v)\\
\end{gathered}
$$
\end{document}